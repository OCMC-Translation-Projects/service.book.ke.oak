\itId{en}{uk}{lash}{rubrical}{Aloud}{
(aloud)
}%
\itId{en}{uk}{lash}{rubrical}{Aloud.colon}{
\itRid{rubrical}{Aloud}:
}%
\itId{en}{uk}{lash}{rubrical}{Or}{
Or
}%
\itId{en}{uk}{lash}{rubrical}{Thrice}{
(x3).
}%
\itId{en}{uk}{lash}{rubrical}{Twice}{
(twice).
}%
\itId{en}{uk}{lash}{rubrical}{NineTimes}{
(x9).
}%
\itId{en}{uk}{lash}{rubrical}{singExapostilarionEo}{
Then the people sing the Exaposteilarion of the resurrection of the first Eothinon.
}%
\itId{en}{uk}{lash}{rubrical}{cl.eu.lichrysbasil.R009}{
The Reader reads the title of the Apostle.
}%
\itId{en}{uk}{lash}{rubrical}{cl.eu.lichrysbasil.R009a}{
The Reading is from the (1st or 2nd) Epistle of St. Paul to the (.......)(or from the Acts of the Apostles), Chapter.... Verse....
}%
\itId{en}{uk}{lash}{rubrical}{TypikaGospel}{
The subdeacon or a reader who has been tonsured by the Bishop reads the Gospel. The subdeacon can read it in front of the Icon of Christ while the reader reads it from the chanters' stands.
}%
\itId{en}{uk}{lash}{rubrical}{cl.eu.lichrysbasil.R010}{
The Reader reads the Apostle, after which:
}%
\itId{en}{uk}{lash}{rubrical}{cl.eu.lichrysbasil.R010a}{
The Reader reads the Gospel, after which:
}%
\itId{en}{uk}{lash}{rubrical}{cl.eu.lichrysbasil.R011}{
The reading is from the Holy Gospel according to \itRid{rubrical}{name.period}
}%
\itId{en}{uk}{lash}{rubrical}{name}{
Ν
}%
\itId{en}{uk}{lash}{rubrical}{name.period}{
\itRid{rubrical}{name}.
}%
\itId{en}{uk}{lash}{rubrical}{typika.rub001}{
This is the form of the Typika in which it is done by the Sub-Deacon or Reader.
}%
\itId{en}{uk}{lash}{rubrical}{typika.rub002}{
This service has to be done by the Sub Deacon or Reader in case the Priest is absent.
}%
\itId{en}{uk}{lash}{rubrical}{typika.rub003a}{
After completing the Morning Prayers and the 6th Hour,
}%
\itId{en}{uk}{lash}{rubrical}{typika.rub003b}{
the Royal Door is opened.
}%
\itId{en}{uk}{lash}{rubrical}{Antiphonon1}{
First Antiphon
}%
\itId{en}{uk}{lash}{rubrical}{Antiphonon2}{
Second Antiphon
}%
\itId{en}{uk}{lash}{rubrical}{Antiphonon3}{
Third Antiphon
}%
\itId{en}{uk}{lash}{rubrical}{SeeBookEnd}{
(Find this and other hymns at the back of the book.)
}%
\itId{en}{uk}{lash}{rubrical}{KontakionSunday}{
Kontakion of mode of the week.
}%
\itId{en}{uk}{lash}{rubrical}{KontakionResurrection}{
The Reader reads the Resurection Kontakion...
}%
\itId{en}{uk}{lash}{rubrical}{Synaxarion}{
The Synaxarion is read.
}%
\itId{en}{uk}{lash}{rubrical}{Canons}{
We then sing the canons of the 7th, 8th and 9th Ode.
}%
\itId{en}{uk}{lash}{rubrical}{Katavasia}{
KATAVASIA
}%
\itId{en}{uk}{lash}{rubrical}{TypikaTheotokion}{
If it is a Feast day (provided it is not on a Saturday), we sing the Theotokion.
}%
\itId{en}{uk}{lash}{rubrical}{VenerateIcon}{
Then the people venerate the Icon of Christ or of the Saint of the day.
}%
\itId{en}{uk}{lash}{rubrical}{RingBell}{
The bell is rung at the start of the Morning Prayers as well as when singing the Hymns to the Mother of God.
}%
