\itId{en}{uk}{lash}{rubrical}{Aloud}{
(aloud)
}%
\itId{en}{uk}{lash}{rubrical}{Aloud.colon}{
\itRid{rubrical}{Aloud}:
}%
\itId{en}{uk}{lash}{rubrical}{Or}{
Or
}%
\itId{en}{uk}{lash}{rubrical}{Thrice}{
(three times).
}%
\itId{en}{uk}{lash}{rubrical}{Twice}{
(twice).
}%
\itId{en}{uk}{lash}{rubrical}{NineTimes}{
(x9).
}%
\itId{en}{uk}{lash}{rubrical}{singExapostilarionEo}{
Then the people sing the Exaposteilarion of the resurrection of the first Eothinon.
}%
\itId{en}{uk}{lash}{rubrical}{cl.eu.lichrysbasil.R009}{
The Reader reads the title of the Apostle.
}%
\itId{en}{uk}{lash}{rubrical}{cl.eu.lichrysbasil.R009a}{
The Reading is from the (1st or 2nd) Epistle of St. Paul to the (.......)(or from the Acts of the Apostles), Chapter.... Verse....
}%
\itId{en}{uk}{lash}{rubrical}{cl.eu.lichrysbasil.R010}{
The Reader reads the Apostle, after which:
}%
\itId{en}{uk}{lash}{rubrical}{cl.eu.lichrysbasil.R011}{
The reading is from the Holy Gospel according to \itRid{rubrical}{name.period}
}%
\itId{en}{uk}{lash}{rubrical}{name}{
Ν
}%
\itId{en}{uk}{lash}{rubrical}{name.period}{
\itRid{rubrical}{name}.
}%
\itId{en}{uk}{lash}{rubrical}{typika.rub001}{
This is the form of the Typika in which it is done by the Sub-Deacon or Reader.
}%
\itId{en}{uk}{lash}{rubrical}{typika.rub002}{
This service has to be done by the Sub Deacon or Reader in case the Priest is absent.
}%
\itId{en}{uk}{lash}{rubrical}{typika.rub003a}{
After completing the Morning Prayers and the 6th Hour,
}%
\itId{en}{uk}{lash}{rubrical}{typika.rub003b}{
the Royal Door is opened.
}%
\itId{en}{uk}{lash}{rubrical}{Antiphonon1}{
First Antiphon
}%
\itId{en}{uk}{lash}{rubrical}{Antiphonon2}{
Second Antiphon
}%
\itId{en}{uk}{lash}{rubrical}{Antiphonon3}{
Third Antiphon
}%
\itId{en}{uk}{lash}{rubrical}{SeeBookEnd}{
(Find this and other hymns at the back of the book.)
}%
